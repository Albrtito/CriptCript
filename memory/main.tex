\documentclass[a4paper,12pt]{report}

% Paquetes necesarios
\usepackage[utf8]{inputenc}
\usepackage[spanish]{babel}
\usepackage{amsmath, amssymb}
\usepackage{graphicx} % Paquete para manejar imágenes
\usepackage{fancyhdr}
\usepackage{hyperref}
\usepackage{lipsum}
\usepackage{tocbibind}
\usepackage{geometry}
\geometry{left=3cm,right=2.5cm,top=2.5cm,bottom=2.5cm}
\usepackage{float}

% Configuración del pie de página
\pagestyle{fancy}
\fancyhf{}
\fancyfoot[C]{\thepage}

\begin{document}
% portada
\begin{titlepage}
    \begin{center}
        \vspace*{1cm} 
        \includegraphics[width=1\textwidth]{./uc3m.jpg}
        
        \vspace*{3cm} 
        
        {\LARGE Criptografía y seguridad informática\\[0.5cm]} 
        {\LARGE G24 - Reporte 1 \\[0.5cm]}
        
        \vspace*{6cm}
        
        \textbf{Javier Martín Pizarro} \\[0.5cm]
        \textbf{Alberto Pascau Sáez} \\[0.5cm]
        \textbf{Raúl Armas Seriña} \\[0.5cm]
        
        \vspace*{1cm}
    \end{center}
\end{titlepage}

% Índice
\tableofcontents
\newpage

% Introducción sin numeración de capítulo
\chapter{Propósito de la aplicación. Estructura interna}

\section{Propósito de la aplicación}

La aplicación simula una página web, levantada en local por el propio usuario, en la que se pueden crear una serie de desafíos (\textit{challenges}) tanto privados como públicos. Dependiendo de qué tipo de sea cada desafío, el flujo interno (de código) será distinto.

\begin{itemize}
    \item \textbf{Desafío público:} cualquier usuario registrado tiene acceso a ellos.
    \item \textbf{Desafío privado:} solamente pueden ser compartidos con un único usuario. El creador del desafío selecciona al otro usuario que será capaz de verlo.
\end{itemize}

El propósito de esta aplicación es generar un sistema informático que cumpla unos requisitos mínimos. Nótese que a medida que la práctica avance, esta lista podrá verse modificada.:\begin{enumerate}
    \item El sistema debe de ser capaz de registrar usuarios y que su información confidencial quede correctamente cifrada.
    \item El sistema debe ser capaz de permitir un inicio de sesión fluido donde se sea capaz de obtener y comparar los datos cifrados de los usuarios de la base de datos con los proporcionados por el cliente.
    \item El sistema debe de ser capaz de crear y recuperar desafíos cuya información pueda o no estar cifrada.
    \item El sistema debe de ser capaz de permitir a un usuario $A$ leer el desafío creado por el usuario $B$ si este se lo ha compartido.
\end{enumerate}

\section{Estructura interna}

Esta aplicación consta de tres partes fundamentales:
\begin{itemize}
    \item Frontend: esencial para la experiencia de usuario. Actua como interfaz entre el cliente y el backend.
    \item Backend: donde se encuentra la API. Todos los mecanismos de cifrado, autenticación, \textit{hasheo} se encuentran en este directorio.
    \item Base de datos: creada usando MariaDB SQL debido a su simplicidad. La base de datos tiene únicamente dos tablas, \textit{users} y \textit{challenges}.
\end{itemize}

\newpage 



\end{document}
