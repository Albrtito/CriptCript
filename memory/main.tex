%!TEX program = xelatex

\documentclass[a4paper,11pt]{article}

% Paquetes necesarios
%\usepackage[utf8]{inputenc}
\usepackage[spanish]{babel}
\usepackage{amsmath, amssymb}
\usepackage{graphicx} % Paquete para manejar imágenes
\usepackage{fancyhdr}
\usepackage{hyperref}
\usepackage{lipsum}
\usepackage{tocbibind}
\usepackage{geometry}
\geometry{left=3cm,right=2.5cm,top=2cm,bottom=2cm}
\usepackage{float}
\usepackage{fontspec}
\usepackage{caption}
\setmainfont{Times New Roman}

% Configuración del pie de página
\pagestyle{fancy}
\fancyhf{}
\fancyfoot[C]{\thepage}

\begin{document}
% portada
\begin{titlepage}
    \begin{center}
        \vspace*{1cm} 
        \includegraphics[width=1\textwidth]{./uc3m.jpg}
        
        \vspace*{3cm} 
        
        {\LARGE Criptografía y seguridad informática\\[0.5cm]} 
        {\LARGE G24 - Entregable 1 \\[0.5cm]}
        
        \vspace*{6cm}
        
        \textbf{Javier Martín Pizarro: 100495861@alumnos.uc3m.es} \\[0.5cm]
        \textbf{Alberto Pascau Sáez: 100495775@alumnos.uc3m.es} \\[0.5cm]
        \textbf{Raúl Armas Seriña: 100495746@alumnos.uc3m.es} \\[0.5cm]
        
        \vspace*{3cm}
        \textbf{GitHub:
        \href{https://github.com/Albrtito/CriptCript.git}{Albrtito/CriptCript.git}}\\[0.5cm]

    \end{center}
\end{titlepage}

% Eliminamos el índice de contenidos
\tableofcontents
\vspace{2cm}
%\newpage


% Introducción sin numeración de capítulo
\section{Propósito de la aplicación. Estructura interna}

\subsection{Propósito de la aplicación}

La aplicación simula una página web, levantada en local por el propio usuario,
en la que se pueden crear desafíos criptográficos(\textit{challenges}) con algoritmos de cifrado
clásico.\\\\
A la hora de crear un desafío cada usuario podrá elegir el algoritmo con el que
cifrar su desafío.El objetivo de otros usuarios será resolver que algoritmo de
cifrado se ha utilizado y el valor de la clave del cifrado; para ayudar en el
descifrado de desafíos la applicación implementará herramientas de
criptoanálisis simples.\\
Cada desafío sera público o privado, dependiendo de la elección que haga su
autor en su creación.
\begin{itemize}
    \item \textbf{Desafío público:} cualquier usuario registrado tiene acceso a ellos.
    \item \textbf{Desafío privado:} solamente pueden ser compartidos con un
        único usuario. El creador del desafío selecciona qué usuario será capaz de verlo.
\end{itemize}

El propósito de esta aplicación es generar un sistema informático que cumpla
unos requisitos mínimos. Nótese que a medida que la práctica avance, esta lista
podrá verse modificada. Los requisitos para esta primera entregá incluyen las
necesidades mínimas de la práctica.
son:\begin{enumerate}
    \item El sistema debe de ser capaz de registrar y autenticar usuarios.
        Guardando su información confidencial correctamente cifrada(contraseñas).$\rightarrow$\textbf{Requisito de confidencialidad}
    \item El sistema debe ser capaz de permitir un inicio de sesión donde se sea capaz de obtener y comparar 
        los datos cifrados de los usuarios de la base de datos con los proporcionados por el cliente.
    \item El sistema debe de ser capaz de guardar y recuperar desafíos usando
        cifrado simétrico/asimétrico. 
        $\rightarrow$ \textbf{Requisito de confidencialidad}
    \item El sistema debe de ser capaz de permitir a un usuario $A$ leer el desafío creado por el usuario $B$ si este se lo ha compartido.
    \item El sistema debe de autenticar que los mensajes mandados por usuarios a la
        base de datos no han sido alterados y son de quien dicen ser $\rightarrow$ \textbf{Requisito de Integridad y
        Autenticidad}
    \item Para cumplir con el requisito 5 el sistema ha de implementar alguna
        forma de MAC o cifrado autenticado
    \item El sistema debe de ser capaz de mostrar en pantalla todos los
        desafíos, privados y públicos, específicos para cada usuario.
\end{enumerate}

\subsection{Estructura interna}

Esta aplicación consta de tres partes fundamentales:
\begin{itemize}
    \item \textbf{Frontend:} esencial para la experiencia de usuario. Actua como interfaz entre el cliente y el backend.
    \item \textbf{Backend:} donde se encuentra la API, implementada usando
        Flask. Todos los mecanismos de cifrado, autenticación y \textit{hasheo} se encuentran en este directorio.
    \item \textbf{Base de datos:} creada usando MariaDB SQL debido a su
        simplicidad. Actualmente la base de datos utiliza 3 tablas
        \textit{users} ,\textit{private\_challenges} y
        \textit{public\_challenges}.El código para la creación de estas tablas se recoge en el
        \hyperref[sec:TablasSQL]{primer anexo}. 
\end{itemize}
La base del backend y de la base de datos han sido tomadas desde el repositorio
Open Source \textbf{backend-builderplate} del alumno y participante en esta
práctica Javier Martín, una iniciativa que permite agilizar y automatizar el
proceso de levantar una API y una base de datos que complemente a una interfaz
de usuario. La estructura del código de este proyecto hereda del repositorio original.%
\footnote{Repositorio original:
\url{https://github.com/jmartinpizarro/backend-builderplate}.}\\

\textbf{El uso de cada parte y su inizialización usando docker se recoge en el
archivo $README.md$ del repositorio, así como una explicación similar a esta de
la estructura del proyecto.}\\

\subsubsection{Frontend}
La carpeta del frontend contiene todo lo relacionado con la visualización de la
web, `html`, `css`, `javascript`. Los estilos (css) y scripts(Javascripts)
tienen sus propias subcarpetas, el html se encuentra directamente bajo la
carpeta frontend debido a que son pocos archivos y es ahí donde se inicializa el
servidor del frontend. \\

Actualmente no hay protección frente a un ataque de búsqueda de urls, cualquiera puede acceder a todo el contenido de la carpeta frontend desde la web.

\subsubsection{Backend}
La carpeta del backend contiene los archivos de `app.py` y `requirementes.txt` que recogen la creación de la api e instalación de los requisitos necesarios en el servidor de backend. El resto del código se encuentra bajo src. Aquí diferenciamos entre tres carpetas:
\begin{itemize}
\item \textbf{mariaDB} $\rightarrow$ Métodos relacionados con la conexión a la DB

\item \textbf{utils} $\rightarrow$ Clases ocupadas de la autenticación, encriptación, generación de claves y hasheado, además del manager ocupado de utilizar todas esas clases para cifrar/descifrar y autenticar mensajes `MessageManager.py`.

\item \textbf{unittest} $\rightarrow$ Collección de test para comprobar que
    todas las clases funcionan correctamente. Actualmente solo se han
    implementado test para el HMAC pero se establecerá como objetivo para la
    segunda entrega crear test para el resto de clases.
\end{itemize}

Finalmente, bajo la carpeta src se encuentran también los archivos
\textit{name\_routes.py} que contienen el routeado para la comunicación de la aplicación
con el frontend\\

De estas tres carpetas utils es la que interesa para evaluar los métodos
criptográficos implementados. Esta carpeta se ha divido según el propósito del
algoritmo que representa cada clase, creando secciones para algoritmos de
\textbf{autenticación, encriptación, generación de claves, cifrado autenticado, generación y verificación de hashes y
cifrados cásicos}. \\
En sta carpeta también podemos encontrar la clase MessageManager; encargada de cifrar y
autenticar los mensajes con el algoritmo correcto. 
\begin{itemize}
    \item \textbf{Todos estos algoritmos criptográficos han sido implementados usando la librería de python cryptography}
\end{itemize}

\vspace{0.5cm}
\fbox{
    \begin{minipage}[t]{0.8\textwidth}
        \textbf{NOTA:} Para la primera entrega solo se utilizan las clases de AESManger
        para cifrar y MACManager para autenticar, no obstante la estructura de 
        clases está ideada para que otros algoritmos también puedan se utilizados, incluso
        algoritmos de cifrado autenticado como fernet.\\
        Haciendouso de esta estructura para la entrega final, el usuario será
        capaz de elegir con que algoritmos cifrar
    \end{minipage}
}

%\subsection{Flujo de información}


\section{Autenticación de usuarios}
\label{sec:autenticacionUsuarios}
Durante el registro de los usuarios, tanto el usuario como la contraseña de este
son \textit{hasheados} e introducidos en la base de datos.Esto asegura que nunca
se guardarán contraseñas o nombres de usuario en texto plano, manteniendo la
confidencialidad de la contraseña y nombre del usuario fuera del alcance de atacantes.
\\
Este método permite que comprobemos la autenticidad de un usuario o
contraseña a traves de una comparación de igualdad con el hash que tenemos
guardado. Sabemos que dado el mismo texto en claro debemos siempre de obtener el
mismo Hash.
$$H(M) = Hash$$

Entonces, a la hora de iniciar sesión o trabajar con los usuarios en la base de datos, deberemos hacer un \textit{hash} para más tarde compararlo. Si dicho \textit{hash} es igual al esperado, podremos trabajar con él, autenticando que es el usuario introducido es el correcto.

Por supuesto, siempre existe el riesgo de las \textbf{colisiones} tal que $H(M) = H'(M) = \text{colisión}$, pero es algo que es complicado que ocurra, ya que las funciones resumen están diseñadas para minimizar estos posibles errores.\\ 

En esta primera entrega se han utilizado las contraseñas hasheadas de los usuarios como claves para el cifrado de mensajes de forma asimétrica, más sobre esto y la generación de claves en \hyperref[sec:generacionUsoDeClaves]{el punto 3.1}.


\subsection{Implementación con SHA256}

La implementación del hasheado de usuarios se ha hecho utilizando la librería
\textit{hashlib} de python. Esta libreria ofrece multitud de funciones hash, de
las que hemos elegido SHA256 debido a que es una de las más seguras
actualmente.\\\\
El código que permite tanto crear como verificar hashes se encuentra en la clase
\textit{HashManager} con direccion
\textit{./backend/src/utils/HashManger.py}.Capturas de las funciones de creación y
verificación se encuentran en el \hyperref[sec:funcionesHash]{segundo anexo}


\section{Cifrado y descifrado de mensajes:AES256}
Para el cifrado y decifrado de mensajes entre el backend y la base de datos se ha implementado un \textbf{cifrado simétrico}, específicamente \textbf{AES256} con el modo CTR.\\
Hemos elegido el modo CTR debido a que:
\begin{itemize}
    \item No require generar padding para el mensaje a cifrar
    \item Solo hace falta una clave y un valor nonce
    \item El valor del nonce no ha de ser privado sino único, permitiendo que se almacene junto al mensaje cifrado y aumentando la seguridad del cifrado
\end{itemize}

Las funciones que implementan AES son dos \textit{encript\_AES y decript\_AES} y se encuentran dentro de la clase \textit{AESManager} bajo \textit{./backend/src/utils/auth/AESManager.py}.Las capturas de estas funciones se pueden ver en el \hyperref[sec:funcionesAES]{cuarto anexo}. Ambas funciones requieren una clave y un valor para el nonce, la obtención de estos valores se explica en la siguiente sección \\\\
Por último, debemos de mencionar el \textbf{método que se utiliza para guardar el valor del nonce}. Sabiendo  que este valor puede ser público lo únimos al texto cifrado una vez realizado el AES, al realizar el descifrado lo primero que hacemos es obtener los primeros 16 bytes del mensaje para usarlos como nonce. 

\subsection{Generación y uso de claves}
\label{sec:generacionUsoDeClaves}
La generación y comprobación de claves se realiza dentro de la clase KeyGen. Actualmente esta clase tiene un único método que genera las claves usando las contraseñas hasheadas de los usuarios. No obstante los hashes son demasiado grandes para ser usados en el AES256 pues son de 64bytes, debido a esto el método de generación de claves de KeyGen parte a la mitad el hash del usuario y lo devuelve como clave. La captura de este método se puede ver en el \hyperref[sec:funcionesKeyGen]{tercer anexo}. \\

Diferenciamos entre dos tipos de claves que se generan ambas a partir del hash de la contraseña de un usuario. 
\begin{itemize}
\item La clave que se utiliza para cifrar desafíos públicos, que es siempre la misma y se obtiene usando el hash del admin.
\item La clave que se utiliza para cifrar desafíos privados, que se obtendrá usando el hash del usuario que crea el desafío.
\end{itemize}


\textbf{Valor del nonce:} Como estamos usando el modo CTR de AES tenemos que generar un valor nonce que cumpla con que con la misma clave nunca se repita el mismo valor nonce. \\ 
Actualmente esto no esta pasando pues el valor se genera de forma aleatoria por el sistema operativo. Esto se debe a que no se ha podido implementar en el tiempo hasta la primera entrega.\\
La implementación usará el número de desafíos totales más uno como valor para el nonce de cada nuevo cifrado. \\ 


\textbf{Futuras implementaciones:}\\ 
La estructura de clases actual está pensada para que las clases que requiran una clave obtengan a través de la clase KeyGen independizando la creación de claves del resto de algoritmos. Para entregas futuras esta clase generará y almacenará claves en una nueva tabla de la base de datos para no depender de las contraseñas de los usuarios y hacer de todo el proceso algo más seguro. 

\section{Autenticación usando MAC}

\newpage
\section{Anexos}
\centering
\subsection{Tablas SQL}
    \label{sec:TablasSQL}
    Código de inicialización del sql
    \begin{figure}[htbp]
        \centering
        % Primera fila de imágenes
        \begin{minipage}[t]{0.48\textwidth} % Tercio de página
            \centering
            \includegraphics[width=\textwidth]{images/privateChallenge.png}
            \captionof{figure}{Tabla de desafíos privados}
        \end{minipage}
        \hfill
        \begin{minipage}[t]{0.48\textwidth} % Tercio de página
            \centering
            \includegraphics[width=\textwidth]{images/publicChallenge.png}
            \captionof{figure}{Tabla de desafíos públicos}
        \end{minipage}
        \vspace{0.5cm}
        % Segunda fila de imagen (centrada)
        \begin{minipage}[t]{0.7\textwidth} 
            \centering
            \includegraphics[width=\textwidth]{images/users.png}
            \captionof{figure}{Tabla de usuarios}
        \end{minipage}
    \end{figure}


\subsection{Funciones Hash}
    \label{sec:funcionesHash}
    \includegraphics[width=0.8\textwidth]{images/HashFunctions.png} 

\subsection{Funciones KeyGen}
    \label{sec:funcionesKeyGen}
    \includegraphics[width=\textwidth]{images/KeyGen.png}
\subsection{Funciones AES}
    Funciones para el cifrado y descrifrado con AES
    \label{sec:funcionesAES}
    \includegraphics[width=\textwidth]{images/encript_AES.png}
    \includegraphics[width=\textwidth]{images/decript_AES.png}
\end{document}
